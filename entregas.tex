\documentclass{beamer}
\usefonttheme{professionalfonts}
\setbeamertemplate{footline}{}
\beamertemplatenavigationsymbolsempty

\usepackage[utf8]{inputenc}
\usepackage{color}
\usepackage{tcolorbox}
\usepackage{newunicodechar}

% https://analyzethedatanotthedrivel.org/2011/08/15/typesetting-utf8-apl-code-with-the-latex-lstlisting-package/
\usepackage{fontspec}
\setmonofont{APL385.ttf}
\usepackage{listings}
\lstdefinestyle{apl}{
  extendedchars=true,
  basicstyle=\ttfamily,
}

\makeatletter
\lst@InputCatcodes
\def\lst@DefEC{%
 \lst@CCECUse \lst@ProcessLetter
  ^^80^^81^^82^^83^^84^^85^^86^^87^^88^^89^^8a^^8b^^8c^^8d^^8e^^8f%
  ^^90^^91^^92^^93^^94^^95^^96^^97^^98^^99^^9a^^9b^^9c^^9d^^9e^^9f%
  ^^a0^^a1^^a2^^a3^^a4^^a5^^a6^^a7^^a8^^a9^^aa^^ab^^ac^^ad^^ae^^af%
  ^^b0^^b1^^b2^^b3^^b4^^b5^^b6^^b7^^b8^^b9^^ba^^bb^^bc^^bd^^be^^bf%
  ^^c0^^c1^^c2^^c3^^c4^^c5^^c6^^c7^^c8^^c9^^ca^^cb^^cc^^cd^^ce^^cf%
  ^^d0^^d1^^d2^^d3^^d4^^d5^^d6^^d7^^d8^^d9^^da^^db^^dc^^dd^^de^^df%
  ^^e0^^e1^^e2^^e3^^e4^^e5^^e6^^e7^^e8^^e9^^ea^^eb^^ec^^ed^^ee^^ef%
  ^^f0^^f1^^f2^^f3^^f4^^f5^^f6^^f7^^f8^^f9^^fa^^fb^^fc^^fd^^fe^^ff%
  ^^^^20ac^^^^0153^^^^0152%
  ^^^^20a7^^^^2190^^^^2191^^^^2192^^^^2193^^^^2206^^^^2207^^^^220a%
  ^^^^2218^^^^2228^^^^2229^^^^222a^^^^2235^^^^223c^^^^2260^^^^2261%
  ^^^^2262^^^^2264^^^^2265^^^^2282^^^^2283^^^^2296^^^^22a2^^^^22a3%
  ^^^^22a4^^^^22a5^^^^22c4^^^^2308^^^^230a^^^^2336^^^^2337^^^^2339%
  ^^^^233b^^^^233d^^^^233f^^^^2340^^^^2342^^^^2347^^^^2348^^^^2349%
  ^^^^234b^^^^234e^^^^2350^^^^2352^^^^2355^^^^2357^^^^2359^^^^235d%
  ^^^^235e^^^^235f^^^^2361^^^^2362^^^^2363^^^^2364^^^^2365^^^^2368%
  ^^^^236a^^^^236b^^^^236c^^^^2371^^^^2372^^^^2373^^^^2374^^^^2375%
  ^^^^2377^^^^2378^^^^237a^^^^2395^^^^25af^^^^25ca^^^^25cb%
  ^^00}
\lst@RestoreCatcodes
\makeatother

\newunicodechar{λ}{\ensuremath\lambda}
\definecolor{c1}{RGB}{219,78,158}
\definecolor{c2}{RGB}{18,110,213}
\definecolor{c3}{RGB}{251,0,29}
\definecolor{c4}{RGB}{114,0,172}
\definecolor{codebackground}{RGB}{234,236,238}

\newcommand{\prim}{\textcolor{c1}}
\newcommand{\secondary}{\textcolor{c2}}
\newcommand{\tertiary}{\textcolor{c3}}
\newcommand{\quarterly}{\textcolor{c4}}

\newtcbox{\enunciado}[1][]{tcbox width=auto limited, colback=purple!5!white,colframe=purple!75!black,boxsep=4pt,boxrule=1pt,#1}

\begin{document}

\begin{frame}[fragile]
\frametitle{λ-calculus - Ejercicio 3.5}

\begin{center}
\enunciado{
    $$ ( λx.( ( λ y.x y ) z ) ) ( λx.x y ) $$
}

$$
( \prim{λx} . ( \secondary{λy} . \prim{x} \: \secondary{y} ) \: z ) \:
( \tertiary{λx} . \tertiary{x} \: y )
$$

\begin{tabular}{l|r}
  Orden Normal & Orden Aplicativo\\ \hline
  $ ( \prim{λx} . ( λ y. \prim{x} y ) z ) \prim{( λx'.x' y')} $ & $ ( λx. (\prim{λy} . x \prim{y} ) \prim{z}) ( λx'.x' y' ) $\\
  $ ( \prim{λy} . ( λx'.x' y' ) \prim{y} ) \prim{z} $           & $ ( \prim{λx} . \prim{x} z ) \prim{(λx'.x' y')} $\\
  $ ( \prim{λx'} . \prim{x'} y' ) \prim{z} $                    & $ ( \prim{λx'} .  \prim{x'} y' ) \prim{z} $\\
  $ z y' $                                                         & $ z y' $\\
\end{tabular}

\end{center}
\end{frame}

\begin{frame}[fragile]
\frametitle{λ-calculus - Ejercicio 3.6}

\begin{center}
\enunciado{
    $$ ( ( λy.( λx.( ( λ x.λy.x ) x ) ) y ) M ) N $$
}

$$
(\prim{λy} . ( \secondary{λx} . ( \tertiary{λxy} . \tertiary{x} ) \secondary{x} )  \prim{y} )  M N
$$

\begin{tabular}{l|r}
  Orden Normal & Orden Aplicativo\\ \hline
  $ (\prim{λy}. (λx. (λx'y'.x')x)\prim{y}) \prim{M} N $ & $ (λy. (λx. (\prim{λx'}y'.\prim{x'})\prim{x})y) M N $ \\
  $ (\prim{λx}. (λx'y'.x')\prim{x}) \prim{M} N  $       & $ (λy. (\prim{λx}y'.\prim{x})\prim{y}) M N $ \\
  $ (\prim{λx'} y' . \prim{x'}) \prim{M} N  $           & $ (\prim{λy}y'.\prim{y}) \prim{M} N $ \\
  $ (\prim{λy'}.M) \prim{N}  $                          & $ (\prim{λy'}.M) \prim{N}  $ \\
  $ M $                                                 & $ M $ \\
\end{tabular}

\end{center}
\end{frame}

\begin{frame}[fragile]
  \frametitle{APL - Ejercicio 4}

  \begin{center}
      \begin{lstlisting}[
        style=apl,
        backgroundcolor=\color{codebackground},
        frame=tbrl,
        xleftmargin=1.5cm,
        xrightmargin=1.5cm]
      A ← (1 + ⍳ 3) , 3 + ⍳ 3
      A [1 4]
  2 4
      A [A]
  3 4 4 4 5 6
      A [A , A]
  3 4 4 4 5 6 3 4 4 4 5 6
      A [⌊A ÷ 2]
  2 2 3 3 3 4
      \end{lstlisting}
  \end{center}
\end{frame}

\begin{frame}[fragile]
  \frametitle{APL - Ejercicio 5}

  \begin{center}
      \begin{lstlisting}[
        style=apl,
        backgroundcolor=\color{codebackground},
        frame=tbrl,
        xleftmargin=0.5cm,
        xrightmargin=0.5cm]
      B ← 'SIC TRANSIT' , 'GLORIA MUNDI'
      ⍴ B
  23
      B [2 × ⍳ 3]
  I R
      B [1 + (⍴ B) - ⍳ ⍴ B]
  IDNUM AIROLGTISNART CIS
      \end{lstlisting}
  \end{center}
\end{frame}

\begin{frame}[fragile]
  \frametitle{APL - Ejercicio 7}

  \begin{center}
      \begin{lstlisting}[
        style=apl,
        backgroundcolor=\color{codebackground},
        frame=tbrl,
        basicstyle=\ttfamily\tiny,
        xleftmargin=0.8cm,
        xrightmargin=0.8cm]
      4 5 ⍴ V ← 2 1 3 2 4 5 6 6 2 1
  2 1 3 2 4
  5 6 6 2 1
  2 1 3 2 4
  5 6 6 2 1
      T ← 3 3 4 ⍴ V
      , T
  2 1 3 2 4 5 6 6 2 1 2 1 3 2 4 5 6 6 2 1 2 1 3 2 4 5 6 6 2 1 2 1 3 2 4 5
      ⍴ T
  3 3 4
      ⍴ , T
  36
      \end{lstlisting}
  \end{center}
\end{frame}

\begin{frame}[fragile]
  \frametitle{APL - Ejercicio 11g}

  \begin{center}

    \enunciado{
      El promedio entre el primer número positivo de un vector V y el último número negativo del mismo.
    }

      \begin{lstlisting}[
        style=apl,
        backgroundcolor=\color{codebackground},
        frame=tbrl,
        xleftmargin=0.8cm,
        xrightmargin=0.8cm]
  AAAA
      \end{lstlisting}
  \end{center}
\end{frame}

\end{document}
